%------------------------------------------------------------
\section{Introduction}
\label{sec-introduction}
NEEDS TO BE UPDATED!
%GCMS intro
As the term suggests, multibody systems consist of several---possibly deformable---bodies, whose relative motion is typically constrained kinematically by means of joints and connectors.
%In multibody dynamics systems, the mechanical part of a machine consists of several moving and possibly deformable bodies. 
Conventional approaches for the analysis of multibody dynamics are based on the floating frame of reference formulation (FFRF)~\cite{Shabana13}, in which a body-fixed frame is employed for the representation of a component's rigid body motion.
In FFRF-based methods, the flexible deformation, which is usually assumed to be small, is introduced relative to the floating frame.
In order to get along with a small number of flexible degrees of freedom, the FFRF is often combined with the ideas of component mode synthesis (CMS). 
Accordingly, the flexible deformation of a component is described by means of appropriate modal shape functions that capture the essential contribution of a component's flexibility.
Due to the flexible deformation being represented relative to the floating frame, the relation between the total displacement and generalized coordinates---regardless of how the rigid body motion is parameterized---is inherently non-linear. 

As opposed to the conventional FFRF-based CMS, it has recently been shown that it is possible to represent the motion of the arbitrarily moving components of complex flexible multibody systems using a linear combination of shape functions and generalized coordinates~\cite{gerstmayr2008,pechstein2013,humerziegler2013}. 
The crucial advantage of this approach, which is referred to as generalized component mode synthesis (GCMS), is the linear relationship between the generalized coordinates and the total displacements.
Due to the linearity, the mass matrix is constant and the stiffness matrix is a co-rotated but otherwise constant matrix that can be computed efficiently. 
Moreover, non-linear inertia terms inherent to conventional FFRF-based approaches do not appear in the GCMS formulation since no relative coordinates are utilized.
These advantages, however, come at cost of a larger set of shape functions as compared to FFRF-based CMS.
The generalized modal basis is constructed from the same mode shapes that are used in FFRF approaches.  

Besides the computationally efficiency of the GCMS, the linear structure of the inertia terms enables a straightforward application of energy-momentum conserving time integration schemes. 
This allows larger time steps for the simulation while preserving exactly fundamental properties of moving structures~\cite{humergerstmayr2013}.
A further feature that plays an important role in the proposed approach is the simplified structure of the constraint relations of typical kinematic pairs compared to FFRF-based methods, which is again due to the immediate correspondence of generalized coordinates and absolute displacement field.

%GMP intro
%The global modal parameterization (GMP), in turn, has been developed for an efficient modeling and simulation of both rigid and flexible multibody systems~\cite{Bruls_IJNME2008, NaetsHeirman2011}. 
%The main idea of GMP is to approximate the configuration of a multibody system by means of system level shape functions.
%%In order to obtain a computationally efficient method, specific shape functions are used for a set of possible configurations. 
%In order to obtain a computationally efficient method, specific shape functions are required for all possible configurations of the multibody system. 
%As the projection space is continuously changing in general, the projection modes are computed and stored for a predefined grid of configurations in a preprocessing step. During the transient analysis, the projection modes are interpolated on that grid of configurations.
%Besides an inevitable error, such an interpolation unfortunately requires a considerable amount of storage.
%%Unfortunately, such interpolation can be a source of errors; moreover, a considerable amount of storage is required.
%%Unfortunately, this leads to an error due to the interpolation and requires considerable storage space to store different spaces.
%This variable projection space also leads to complicated inertial forces, which greatly increases the computational load. 
%For these reasons, it is beneficial to have both a constant reduction space and a constant reduced mass-matrix, which is enabled by combining the fundamental ideas and advantages of the GCMS and GMP approach. 
Introduction on system-level MOR for flexible multibody simulation:
\begin{itemize}
\item Interest in system-level MOR as the component based description for fmbs is typically redundant.
\item One method which has been introduced in the past as a general approach to tackle this issue, is the GMP approach. But highly nonlinear inertial forces, poor scaling with the number of rigid DOFs and large storage requirements.
\item Focus to resolve these issues by starting from the GCMS model description specifically. 
\end{itemize}
ADD A PARAGRAPH ON MOR TECHNIQUES WHICH COULD BE APPLIED TO THE SYSTEM LEVEL MODEL: GMP, (U)DEIM, ECSW AND THEIR DRAWBACKS FOR THIS PARTICULAR APPLICATION. 

%combined method intro:
%In the present paper, we propose a coupled approach, in which the GCMS formulation for the individual components of a multibody system serves as basis for a GMP model reduction. 
%Coupling these two approaches leads to a description with a constant mass matrix, from which a number of the redundant degrees of freedom (DOFs) introduced by the GCMS approach and the kinematic constraints that connect the bodies can be eliminated. 
%In the case of linear constraint equations, which comprises a large range of kinematic pairs due to the specific nature of generalized coordinates in the GCMS, a constant joint projection can be defined. 
%This characteristic feature is exploited for obtaining a constant reduction space based on a singular value decomposition (SVD) of the undeformed configurations and linearized flexible deformations. 
%The SVD leads to a dominant space which captures the essential motion of the full system.
%Such description enables highly efficient simulation of complex flexible multibody systems. 
Two important steps in the proposed system level model reduction:
\begin{itemize}
\item Construct reduction space. Focus on constant reduction space in order to circumvent nonlinear inertial forces and issues with model interpolation. Two different approaches, an a-priori and posterior one, are presented in this work for constructing the reduction space. 
\item Efficient evalution of nonlinear internal forces. Circumvent the need to sample the internal force space. 
\end{itemize}

The construction of this system level reduced model occurs in multiple steps:
\begin{itemize}
\item First the original GCMS model is constructed from (solid) finite-element models for all the system components.
\item In the second step orthogonal space to the constraint equations is evaluated.
\item In the third step the system-level reduction space is constructed.
\item In the fourth step the different components which can be precomputed for the reduced system level evaluation are constructed.
\item In the final step the reduced system level model is evaluated. 
\end{itemize}
This leads to the system-level mode synthesis (SLMS) proposed in this work. 

In Section~\ref{sec-gcms}, we briefly outline the key ideas and properties of the GCMS formulation. 
Section~\ref{sec-GMP_overview} discusses how the GCMS approach is particularly attractive from a system level model reduction point of view and how this formulation can be used to obtain an ordinary differential description of an MBD model. Section~\ref{sec-reduction_space} then describes how the system level reduction space can be constructed and Section~\ref{sec-hyper_reduction} presents an efficient way for evaluating the nonlinear internal forces for a system level model. 
In order to demonstrate the potential of the proposed approach, two numerical examples are discussed in Section~\ref{sec-numerical_example}.

