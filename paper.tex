%%%%%%%%%%%%%%%%%%%%%%% file template.tex %%%%%%%%%%%%%%%%%%%%%%%%%
%
% This is a general template file for the LaTeX package SVJour3
% for Springer journals.          Springer Heidelberg 2010/09/16
%
% Copy it to a new file with a new name and use it as the basis
% for your article. Delete % signs as needed.
%
% This template includes a few options for different layouts and
% content for various journals. Please consult a previous issue of
% your journal as needed.
%
%%%%%%%%%%%%%%%%%%%%%%%%%%%%%%%%%%%%%%%%%%%%%%%%%%%%%%%%%%%%%%%%%%%
%
%\documentclass{svjour3}                     % onecolumn (standard format)
%\documentclass[smallcondensed]{svjour3}     % onecolumn (ditto)
\documentclass[smallextended]{svjour3}       % onecolumn (second format)
%\documentclass[twocolumn]{svjour3}          % twocolumn
%
\smartqed  % flush right qed marks, e.g. at end of proof
%
\usepackage{graphicx}
\usepackage[applemac]{inputenc}
\usepackage{amsmath,amssymb}
\usepackage{bm}
\usepackage{booktabs}
\usepackage[numbers,sort&compress]{natbib}
\usepackage{mathptmx}      % use Times fonts if available on your TeX system
\usepackage{algorithmic}
\usepackage{algorithm}
%
% insert here the call for the packages your document requires
%\usepackage{latexsym}
% etc.
%
% please place your own definitions here and don't use \def but
% \newcommand{}{}
%
% Insert the name of "your journal" with

\newcommand{\be}{\begin{equation}}
\newcommand{\ee}{\end{equation}}


\newcommand{\ex}{\mathbf e_1}
\newcommand{\ey}{\mathbf e_2}
\newcommand{\ez}{\mathbf e_3}
\newcommand{\ej}{\mathbf e_j}
\renewcommand{\r}{\mathbf r}
\newcommand{\rr}{\mathbf r_{R}}
\newcommand{\uf}{\mathbf u_{f}}
\newcommand{\ut}{\mathbf u_{t}}
\newcommand{\ur}{\mathbf u_{r}}
\newcommand{\A}{\mathbf A}
\newcommand{\x}{\mathbf x}
\renewcommand{\u}{\mathbf u}
\renewcommand{\v}{\mathbf v}
\newcommand{\I}{\mathbf I}
\newcommand{\E}{\mathbf E}
\newcommand{\gradu}{\nabla \u}
\newcommand{\graduf}{\gradu_f}
\newcommand{\Elin}{\widetilde{\E}}
\DeclareMathOperator{\sym}{sym}
\newcommand{\dv}{\, \mathrm dV}
\renewcommand{\S}{\mathbf S}
\newcommand{\D}{\mathbf D}
\newcommand{\fext}{\mathbf f_{\mathrm{ext}}}
\renewcommand{\t}{\mathbf t}
\newcommand{\q}{\mathbf q}
\newcommand{\qgcms}{\q^{\rm GCMS}}
\newcommand{\dqgcms}{\dot{\q}^{\rm GCMS}}
\newcommand{\ddqgcms}{\ddot{\q}^{\rm GCMS}}
\newcommand{\vgcms}{\v^{\rm GCMS}}
\newcommand{\qgcmspp}{\ddot{\q}^{\rm GCMS}}
\newcommand{\N}{\mathbf N}
\newcommand{\Ngcms}{\N^{\rm GCMS}}
\newcommand{\M}{\mathbf M}
\newcommand{\K}{\mathbf K}
\newcommand{\qrigid}{\q_{\rm rigid}^{\rm GCMS}}
\newcommand{\qflex}{\q_{\rm flex}^{\rm GCMS}}
\newcommand{\Mgcms}{\M^{\rm GCMS}}
\newcommand{\Kgcms}{\mathbf K^{\rm GCMS}}
\newcommand{\fnl}{\mathbf f_{\rm nl}}
\newcommand{\C}{\mathbf C}
\renewcommand{\l}{\bm \lambda}
\newcommand{\phigcms}{\bm \Phi^{\rm GCMS}}
\newcommand{\Abd}{\mathbf A_{\mathrm{bd}}}
\newcommand{\At}{\mathbf A_{\mathrm{t}}}
\newcommand{\Kred}{\mathbf K_{\rm red}}
\renewcommand{\L}{\mathbf L}
\newcommand{\J}{\mathbf J}


\newcommand{\Kfe}{\mathbf K^{\rm FE}}
\newcommand{\Mfe}{\mathbf M^{\rm FE}}

\newcommand{\Vstrain}{V_{\mathrm{strain}}}
\newcommand{\Eint}{E_{\mathrm{int}}}
\newcommand{\Einer}{E_{\mathrm{iner}}}
\newcommand{\Wext}{W_{\mathrm{ext}}}
\newcommand{\Fint}{F_{\mathrm{int}}}
\newcommand{\Finer}{F_{\mathrm{iner}}}
\newcommand{\Fext}{F_{\mathrm{ext}}}

\newcommand{\realn}{\mathbb{R}^{n}}

\newcommand{\kernel}{\mathbf N_0}
\newcommand{\qgmp}{\q^{\rm GMP}}
\newcommand{\qgmpp}{\dot{\q}^{\rm GMP}}
\newcommand{\qgmppp}{\ddot{\q}^{\rm GMP}}

\newcommand{\thetagmp}{\bm \theta}
\newcommand{\rhogmp}{\bm \rho (\thetagmp)}
\newcommand{\phigmp}{\bm \Phi^{\rm GMP}}
%\newcommand{\phigmp}{\bm \Phi^{\rm GMP}}
\newcommand{\deltagmp}{\bm \delta}
\newcommand{\etagmp}{\bm \eta}
\newcommand{\Mgmp}{\M^{\rm GMP}}
\newcommand{\Kgmp}{\K^{\rm GMP}}

\newcommand{\qsl}{\q^{\rm sl}}
\newcommand{\qslp}{\dot{\q}^{\rm sl}}
\newcommand{\qslpp}{\ddot{\q}^{\rm sl}}

\newcommand{\thetasl}{\bm \theta}
\newcommand{\rhosl}{\bm \rho }
\newcommand{\phisl}{\bm \Phi^{\rm sl}}
%\newcommand{\phigmp}{\bm \Phi^{\rm GMP}}
\newcommand{\deltasl}{\bm \delta}
\newcommand{\etasl}{\bm \eta}
\newcommand{\detasl}{\dot{\bm \eta}}
\newcommand{\ddetasl}{\ddot{\bm \eta}}
\newcommand{\Msl}{\M_{\rm sl}}
\newcommand{\Ksl}{\K_{\rm sl}}


\journalname{Multibody System Dynamics}
%
\begin{document}

%\title{A generalized component mode synthesis approach for global modal parameterization in flexible multibody dynamics%\thanks{Grants or other notes
\title{System level model reduction for flexible multibody simulation using generalized component mode synthesis}
%about the article that should go on the front page should be
%placed here. General acknowledgments should be placed at the end of the article.}
%\subtitle{Do you have a subtitle?\\ If so, write it here}

\titlerunning{System-Level Global Component Mode Synthesis}        % if too long for running head

\author{Frank Naets \and 
	Alexander Humer \and \\
	Wim Desmet \and
	Johannes Gerstmayr
}

%\authorrunning{Short form of author list} % if too long for running head

\institute{A. Humer \at
              Institute of Technical Mechanics, Johannes Kepler University,  Altenberger Stra�e 69, 4040 Linz, Austria \\
%              Tel.: +123-45-678910\\
%              Fax: +123-45-678910\\
              \email{alexander.humer@jku.at}           %  \\
%             \emph{Present address:} of F. Author  %  if needed
           \and
           F. Naets \and W. Desmet \at
		Department of Mechanical Engineering, KU Leuven, Celestijnenlaan 300, 3001 Heverlee, Belgium \\
	\email{frank.naets@kuleuven.be, wim.desmet@kuleuven.be}
	\and
	J. Gerstmayr \at 
	Institute of Mechatronics, University of Innsbruck, Technikerstra�e 13, 6020 Innsbruck, Austria \\
	\email{johannes.gerstmayr@uibk.ac.at}	
}

\date{Received: date / Accepted: date}
% The correct dates will be entered by the editor


\maketitle

\begin{abstract}
NEEDS TO BE UPDATED!
Model-order reduction is a common way to efficiently investigate the dynamic behavior of complex structures and machines in real world applications. % or even in real-time.
%
Recently, a novel approach denoted as a generalized component mode synthesis (GCMS) has been developed, which is based on a modal reduction for moving and rotating flexible bodies. 
The defining property of GCMS is a linear configuration space, which is obtained at the expense of a larger set of coordinates. % compared to the conventional floating frame of reference formulation (FFRF) based component mode synthesis (CMS). 
The simple and uniform structure of the equations of motion, which is built upon a specific co-rotational strain measure, leads to linear inertia terms and allows, e.g., a straight-forward application of energy-momentum consistent time-integration schemes.
%

A conceptually different approach for the modelling of rigid and flexible multibody dynamics systems is the global modal parameterization (GMP). In this method, the global motion and deformation of several bodies is interpolated by a set of global shape functions that may change over time.
Specific formulations of GMP attempt to achieve a computationally efficient representation as well as an optimal interpolation of such time varying global shape functions.
%
In order to construct a global modal parameterization, a constraint reduction is typically performed. 
As a consequence, strongly nonlinear terms occur in the equations of motion and, in particular, the non-constant mass matrix and gyroscopic terms entail significant computational costs. 
Not least for these reasons, it is beneficial to have a constant reduction space providing a constant reduced mass-matrix, which is enabled by combining the fundamental ideas and advantages of the GCMS and the GMP approach. 
%
In the present paper, the fundamental ideas of GCMS and GMP are shortly reviewed, and the combined method is presented. 
A numerical example of a modally reduced flexible multibody system demonstrating the efficiency of the combined method is shown.
\keywords{Model-order reduction \and Modal projection \and Flexible multibody systems \and Dynamics}
% \PACS{PACS code1 \and PACS code2 \and more}
% \subclass{MSC code1 \and MSC code2 \and more}
\end{abstract}

%------------------------------------------------------------
\section{Introduction}
\label{sec-introduction}
NEEDS TO BE UPDATED!
%GCMS intro
As the term suggests, multibody systems consist of several---possibly deformable---bodies, whose relative motion is typically constrained kinematically by means of joints and connectors.
%In multibody dynamics systems, the mechanical part of a machine consists of several moving and possibly deformable bodies. 
Conventional approaches for the analysis of multibody dynamics are based on the floating frame of reference formulation (FFRF)~\cite{Shabana13}, in which a body-fixed frame is employed for the representation of a component's rigid body motion.
In FFRF-based methods, the flexible deformation, which is usually assumed to be small, is introduced relative to the floating frame.
In order to get along with a small number of flexible degrees of freedom, the FFRF is often combined with the ideas of component mode synthesis (CMS). 
Accordingly, the flexible deformation of a component is described by means of appropriate modal shape functions that capture the essential contribution of a component's flexibility.
Due to the flexible deformation being represented relative to the floating frame, the relation between the total displacement and generalized coordinates---regardless of how the rigid body motion is parameterized---is inherently non-linear. 

As opposed to the conventional FFRF-based CMS, it has recently been shown that it is possible to represent the motion of the arbitrarily moving components of complex flexible multibody systems using a linear combination of shape functions and generalized coordinates~\cite{gerstmayr2008,pechstein2013,humerziegler2013}. 
The crucial advantage of this approach, which is referred to as generalized component mode synthesis (GCMS), is the linear relationship between the generalized coordinates and the total displacements.
Due to the linearity, the mass matrix is constant and the stiffness matrix is a co-rotated but otherwise constant matrix that can be computed efficiently. 
Moreover, non-linear inertia terms inherent to conventional FFRF-based approaches do not appear in the GCMS formulation since no relative coordinates are utilized.
These advantages, however, come at cost of a larger set of shape functions as compared to FFRF-based CMS.
The generalized modal basis is constructed from the same mode shapes that are used in FFRF approaches.  

Besides the computationally efficiency of the GCMS, the linear structure of the inertia terms enables a straightforward application of energy-momentum conserving time integration schemes. 
This allows larger time steps for the simulation while preserving exactly fundamental properties of moving structures~\cite{humergerstmayr2013}.
A further feature that plays an important role in the proposed approach is the simplified structure of the constraint relations of typical kinematic pairs compared to FFRF-based methods, which is again due to the immediate correspondence of generalized coordinates and absolute displacement field.

%GMP intro
%The global modal parameterization (GMP), in turn, has been developed for an efficient modeling and simulation of both rigid and flexible multibody systems~\cite{Bruls_IJNME2008, NaetsHeirman2011}. 
%The main idea of GMP is to approximate the configuration of a multibody system by means of system level shape functions.
%%In order to obtain a computationally efficient method, specific shape functions are used for a set of possible configurations. 
%In order to obtain a computationally efficient method, specific shape functions are required for all possible configurations of the multibody system. 
%As the projection space is continuously changing in general, the projection modes are computed and stored for a predefined grid of configurations in a preprocessing step. During the transient analysis, the projection modes are interpolated on that grid of configurations.
%Besides an inevitable error, such an interpolation unfortunately requires a considerable amount of storage.
%%Unfortunately, such interpolation can be a source of errors; moreover, a considerable amount of storage is required.
%%Unfortunately, this leads to an error due to the interpolation and requires considerable storage space to store different spaces.
%This variable projection space also leads to complicated inertial forces, which greatly increases the computational load. 
%For these reasons, it is beneficial to have both a constant reduction space and a constant reduced mass-matrix, which is enabled by combining the fundamental ideas and advantages of the GCMS and GMP approach. 
Introduction on system-level MOR for flexible multibody simulation:
\begin{itemize}
\item Interest in system-level MOR as the component based description for fmbs is typically redundant.
\item One method which has been introduced in the past as a general approach to tackle this issue, is the GMP approach. But highly nonlinear inertial forces, poor scaling with the number of rigid DOFs and large storage requirements.
\item Focus to resolve these issues by starting from the GCMS model description specifically. 
\end{itemize}
ADD A PARAGRAPH ON MOR TECHNIQUES WHICH COULD BE APPLIED TO THE SYSTEM LEVEL MODEL: GMP, (U)DEIM, ECSW AND THEIR DRAWBACKS FOR THIS PARTICULAR APPLICATION. 

%combined method intro:
%In the present paper, we propose a coupled approach, in which the GCMS formulation for the individual components of a multibody system serves as basis for a GMP model reduction. 
%Coupling these two approaches leads to a description with a constant mass matrix, from which a number of the redundant degrees of freedom (DOFs) introduced by the GCMS approach and the kinematic constraints that connect the bodies can be eliminated. 
%In the case of linear constraint equations, which comprises a large range of kinematic pairs due to the specific nature of generalized coordinates in the GCMS, a constant joint projection can be defined. 
%This characteristic feature is exploited for obtaining a constant reduction space based on a singular value decomposition (SVD) of the undeformed configurations and linearized flexible deformations. 
%The SVD leads to a dominant space which captures the essential motion of the full system.
%Such description enables highly efficient simulation of complex flexible multibody systems. 
Two important steps in the proposed system level model reduction:
\begin{itemize}
\item Construct reduction space. Focus on constant reduction space in order to circumvent nonlinear inertial forces and issues with model interpolation. Two different approaches, an a-priori and posterior one, are presented in this work for constructing the reduction space. 
\item Efficient evalution of nonlinear internal forces. Circumvent the need to sample the internal force space. 
\end{itemize}

The construction of this system level reduced model occurs in multiple steps:
\begin{itemize}
\item First the original GCMS model is constructed from (solid) finite-element models for all the system components.
\item In the second step orthogonal space to the constraint equations is evaluated.
\item In the third step the system-level reduction space is constructed.
\item In the fourth step the different components which can be precomputed for the reduced system level evaluation are constructed.
\item In the final step the reduced system level model is evaluated. 
\end{itemize}
This leads to the system-level mode synthesis (SLMS) proposed in this work. 

In Section~\ref{sec-gcms}, we briefly outline the key ideas and properties of the GCMS formulation. 
Section~\ref{sec-GMP_overview} discusses how the GCMS approach is particularly attractive from a system level model reduction point of view and how this formulation can be used to obtain an ordinary differential description of an MBD model. Section~\ref{sec-reduction_space} then describes how the system level reduction space can be constructed and Section~\ref{sec-hyper_reduction} presents an efficient way for evaluating the nonlinear internal forces for a system level model. 
In order to demonstrate the potential of the proposed approach, two numerical examples are discussed in Section~\ref{sec-numerical_example}.


%------------------------------------------------------------
% Section on GCMS: Alex + Johannes
\section{{Generalized component mode synthesis}} \label{sec-gcms}
%++++++++++++++++++++++++++++++++++++++++
%some text here still copy/paste from ASME paper!!!
%++++++++++++++++++++++++++++++++++++++++

% I can probably keep this section as-is...

%General overview of GCMS (high level). 
%Current bottlenecks...
%\subsection{Modal projection}
%Explain 12+9Nm modes.
%\subsection{Equations of motion}
%Show reduced equations of motion. 

The generalized component mode synthesis (GCMS) is a reduction method for flexible multibody systems by means of a formulation based on inertial or absolute coordinates \cite{gerstmayr2006}. 
Such a formulation, which is also denoted as absolute coordinate formulation (ACF), can be regarded as a method dual to the FFRF in the sense that the complete set of coordinates is defined in the global frame rather than employing coordinates relative to a co-rotational frame.
%In what follows, the formulation is shortly revisited.

\subsection{Absolute coordinate formulation and modal reduction}

Following Pechstein et al.~\cite{pechstein2013}, we introduce a modal reduction of some coordinates~$\q$, e.g., nodal displacements of solid finite element models,
\be
  \q \approx \phigcms \qgcms. 
\ee
To enable a computationally efficient analysis, the reduced set of coordinates $\qgcms$ needs to be of much smaller dimensionality as compared to the original coordinates~$\q$ but still describe the body's dynamic behavior sufficiently accurate.
%The reduction matrix $\phigcms$ is obtained from appropriate mode shapes, e.g., body-local eigenmodes or static modes, by a particular transformation that accounts for the body's rigid body motion. 
%For further details on the structure of~$\phigcms$, we again refer to Pechstein et al.~\cite{pechstein2013}.
We emphasize the crucial idea of the reduction being constructed such that the total displacement $\u$ of any point $\x$ of a body is linearly related to the reduced coordinates $\qgcms$ via shape functions $\Ngcms$:
\begin{equation} \label{eq:lin}
\u(\x) = \Ngcms (\x) \qgcms .
\end{equation}
In the GCMS formulation, one single set of coordinates $\qgcms$ describes the entire deformation of a body, i.e., rigid body translation and rotation as well as flexible deformation. 
Such an approach differs significantly from the classical FFRF-based CMS method, which uses a different set of coordinates for the rigid body motion and the flexible part of displacements.

Applying modal reduction to the ACF, the reduction matrix $\phigcms$ is obtained from appropriate mode shapes, e.g., body-local eigenmodes or static modes, by a transformation that accounts for the rotation of a body. 
Each standard mode shape is converted into nine corresponding generalized mode shapes.
%A modal reduction method can be applied to the absolute coordinate formulation by transforming each standard mode shape into nine corresponding generalized mode shapes. 
These nine shape functions---resulting from the nine components of the rotation tensor---relate nine associated generalized coordinates to the small flexible displacements of an arbitrarily rotated body.
The underlying rigid body motion, represented by a rotation tensor~$\A$ and a rigid body translation $\ut$ is used for simplifying the computation of the elastic forces of the body.
The rotation tensor $\A$ is obtained by orthogonalization of position vectors of three referential nodes in the underlying finite element mesh. The reduction matrix can then be written as:
\begin{equation}
\phigcms = \begin{bmatrix} \phigcms_t & \phigcms_r & \phigcms_f\end{bmatrix},
\label{eq:phigcms}
\end{equation}
where $\phigcms_t$ are three translation modes for a body, $\phigcms_r$ are nine rotational modes and $\phigcms_f$ are the flexible deformation modes. 
For more details on the construction of reduction matrix $\phigcms$, which links the finite element shape functions $\mathbf N^{\rm FE}(\x)$ to those of the GCMS method, $\Ngcms(\x) = \mathbf N^{\rm FE}(\x) \phigcms$, we again refer to Pechstein et al.~\cite{pechstein2013}.

Remarkably, the coordinates that describe the rigid body rotation of each body also contain stretch and shear deformation if no further constraints are imposed.
A coordinate vector $\qrigid$ needs to be computed, which represents both the translational part and the rotation $\ur = \A \x$ contained in the total displacement, i.e.,
\begin{equation}
\ut(\x) + \ur(\x) = \Ngcms(\x) \qrigid .
\end{equation}
Consequently, the coordinate vector corresponding to the flexible displacements $\uf$ is immediately obtained as the difference $\qflex = \qgcms - \qrigid$ such that
\begin{equation}
\uf(\x) = \Ngcms (\x) \qflex .
\end{equation}
Having the relation of displacements and coordinates at hand enables a continuum mechanics based derivation of the equations of motion, see \cite{gerstmayr2006, pechstein2013}.

%
%++++++++++++++++++++++++++++++++++++++++++++++++++++++++++
\subsection{Modally reduced equations of motion}

The modally reduced equations of motion are derived by starting from a Langrangian description of the system (semi-disretized in space):
\begin{equation}
\mathcal{L} = \Einer - \Eint + \lambda^T\C - \Wext,
\end{equation}
where $\Einer$ represents the kinetic energy, $\Eint$ the internal potential strain energy, $\C$ is a vector with the constraint equations and $\lambda$ are the corresponding Lagrange multipliers and $\Wext$ is the work performed by external forces. The equations of motion are then obtained by applying Hamilton's principle for the GCMS DOFs and Lagrange multipliers:
\begin{eqnarray}
\frac{d}{dt}\left(\frac{\partial\mathcal{L}}{\partial \dqgcms}\right) - \frac{\partial\mathcal{L}}{\partial \qgcms} &= 0, \\
\C & = 0. 
\end{eqnarray}
This can be rewritten as (assuming only holonomic constraints):
\begin{eqnarray}
\Finer + \Fint + \left(\frac{\partial \C}{\partial \qgcms}\right)^T\lambda &= \Fext, \\
\C & = 0. 
\end{eqnarray}
In the following paragraph we briefly summarize how the different contributions in this equation are computed for a GCMS model description.

The inertial forces are easy to obtain (in contrast to a FFR model approach) due to the global nature of the DOFs, as discussed in previous works \cite{pechstein2013} the inertial forces $\Finer$ are obtained as:
\begin{equation}
\Finer = {\phigcms}^T\Mfe\phigcms\ddqgcms = \Mgcms\ddqgcms.
\end{equation}
External forces defined in the global reference frame are equally straightforward to project:
\begin{equation}
\Fext = {\phigcms}^T\Fext^{FE}. 
\end{equation}

However, in the case of the GCMS model description, the internal forces $\Fint$ are relatively involved. For this approach the internal potential strain energy $\Eint$ is obtained as:
\begin{equation}
  \Eint = \frac{1}{2} \left(\At\phigcms\qgcms -u_0\right)^T\Kfe\left(\At\phigcms\qgcms -u_0\right) ,
	\label{eq:w_int2}
\end{equation}
where the stiffness matrix $\Kfe$ is the original finite element stiffness matrix, $u_0$ are the undeformed coordinates expressed in the body axis system and $\At$ denotes a block-diagonal matrix of the size $n \times n$with the three-by-three rotation matrix $\A$ on its diagonal. This rotation matrix can be determined in a number of ways, analogously to the techniques developed for corotational finite elements \cite{Felippa}.  Due to the construction of the of the GCMS reduction space, it is possible to write \cite{gerstmayr2006}:
\begin{equation}
 \At\phigcms\qgcms = \phigcms\Abd\qgcms,
\end{equation}
with $\Abd$ a block-diagonal matrix of the reduced model size $n_{GCMS} \times n_{GCMS}$ with three-by-three rotation matrices $\A$ on its diagonal. The internal energy can then be written as:
\begin{eqnarray}
  \Eint &= \frac{1}{2} \left(\phigcms\Abd\qgcms -u_0\right)^T\Kfe\left(\phigcms\Abd\qgcms -u_0\right) , \\
	&= \frac{1}{2} \left(\phigcms\Abd\qgcms\right)^T\Kfe\left(\phigcms\Abd\qgcms\right) \nonumber \\
	& \qquad - \left(\phigcms\Abd\qgcms\right)^T\Kfe u_0+\frac{1}{2}u_0^T\Kfe u_0 ,\\
	&= \frac{1}{2} \left(\Abd\qgcms\right)^T\Kred\left(\Abd\qgcms\right) - \left(\Abd\qgcms\right)^T\Kred^*+\Eint^0,
\end{eqnarray}
with 
\begin{eqnarray}
\Kred &=& \left(\phigcms\right)^T\Kfe\phigcms, \label{eq:Kred}\\
\Kred^* &=& \left(\phigcms\right)^T\Kfe u_0. \label{eq:Kred*}
\end{eqnarray}
From this internal energy the internal forces acting on the GCMS DOFs can be computed as:
\begin{eqnarray}
\Fint & = & \frac{\partial \Eint}{\partial \qgcms}, \\
&=& \Abd^T\left(\Kred\Abd\qgcms-\Kred^*\right) \nonumber \\
&& + \left[\left(\frac{\partial\Abd}{\partial\qgcms_i}\qgcms\right)^T\left(\Kred\Abd\qgcms-\Kred^*\right)\right]_{i=1:n_{GCMS}}
\end{eqnarray}
with the notation:
\begin{equation}
\left[\frac{\partial f}{\partial q_i}K\right]_{i=1:n} = \begin{bmatrix} \frac{\partial f}{\partial q_1}K \\ \ldots \\ \frac{\partial f}{\partial q_n}K\end{bmatrix}. 
\end{equation}
Upon a closer inspection, one recognizes that the evaluation of these equations requires little computational expense, see \cite{pechstein2013} for further details concerning the implementation.
Although the number of degrees of freedom considerably exceeds that of the corresponding FFRF-based CMS method, the efficiency of the methods have proven to be comparable (or better) in industrial applications~\cite{pechstein2013}.
While the FFRF-based CMS method features a smaller number of generalized coordinates using the same number of flexible mode shapes, a non-constant mass matrix and gyroscopic inertial need to be evaluated in each computation step, which usually negates the advantage obtained by the smaller system size.

The treatment of the joint equations is treated in more detail in the following section. 


% -----------------------------------------------------------
% Section on GMP: Frank
\section{{System level model description}}
\label{sec-GMP_overview}

% Focus on benefits of GCMS for system level MOR, no discussion of GMP anymore...
This section discusses how the equations of motion for a GCMS model can be described on a system level instead of a component level. 
Classical model reduction techniques for the simulation of flexible multibody systems are realized on a component level, i.e., the individual components are reduced separately before the system is eventually assembled.
In contrast to such approaches, which include both the conventional FFRF-based CMS and the GCMS, this work aims to directly cover the full system behavior in a reduced setting. This approach has as main benefit that redundancies which are typically present for regular FMBS can be reduced or eliminated. This can be obtained by employing system level modes instead of component level modes.
This concept was for introduced in the Global Modal Parameterization (GMP)~\cite{Bruls_IJNME2008, NaetsHeirman2011, Naets_MUBO2011}for FMBS. However this approach faces multiple difficulties (as discussed in Section~\ref{sec-introduction}) which can be circumvented by the GCMS based system reduction proposed in this work. 

\subsection{System-level equations of motion} 
In order to obtain an additional reduction of the number of DOFs, the proposed approach relies on two important properties of these system modes:
\begin{enumerate}
\item A smaller number of DOFs is required for the same range of dynamic phenomena. 
\item The modes are defined to meet the constraints such that these can be eliminated from the equations of motion. 
\end{enumerate}
The first requirement is obviously the main idea of model reduction. The second requirement implies that the constraints can be eliminated from the equations of motion, turning the system of differential-algebraic equations into a set of ordinary differential equations.

The generalized coordinates $\q$ describing the GCMS system's configuration are introduced as a the sum of a rigid body configuration and a linear motion field,
\begin{equation}
\q = \rhosl + \phisl \etasl.
\end{equation}
In the above equation, $\rhosl \in \realn$ denotes an undeformed reference configuration of the system which complies with all the joints imposed on the system, $\phisl \in \mathbb{R}^{n\times n_{\eta}}$ is a constant system level motion space (including both rigid and flexible deformation modes) and $\etasl\in \mathbb{R}^{n_{\eta}}$ contains the corresponding participation factors. In contrast to the previously presented GMP approach, both $\rhosl$ and $\phisl$ remain constant over the full motion of the system. 

From this linear motion space, the system level reduced equations of motion can be summarized as:
\begin{eqnarray} \label{eq_eom-gmp1}
\Msl \ddetasl + {\phisl}^T\Fint(\rhosl + \phisl \etasl) - {\phisl}^T\frac{\partial C}{\partial \q}^T\lambda= {\phisl}^T\Fext, \\
C(\rhosl + \phisl \etasl) = 0
\end{eqnarray}
As mentioned above, the algebraic constraints representing the kinematic pairs of the original system will be eliminated from the equations of motion by an appropriate choice of motion shapes $\phisl$ such that only a minimal set of DOFs is employed and the above equations can be further reduced to:
\begin{equation} \label{eq_eom-gmp}
\Msl \ddetasl + {\phisl}^T\Fint(\rhosl + \phisl \etasl) = {\phisl}^T\Fext.
\end{equation}
This choice further reduces the number of variables for which the equations of motion need to be solved and turns them into ordinary differential equations instead of differential-algebraic equations, which are considerably easier to integrate over time and allow more flexibility with respect to solver choice. 

\subsection{Constraint elimination}
\label{sec:constraint_elimination}
In order to obtain Eq~\eqref{eq_eom-gmp}, the motion projection space $\phisl$ should lie in the kernel (or null) space $\kernel$ of the constraints $\C(\q)$,
\begin{equation}
\kernel = ker(\C) = \left\{\q \in \realn , \C(\q) = \mathbf{0} \right\}.
\end{equation}
The GCMS description inherently leads to linear equations for several fundamental joints, in contrast to a FFR description in which the joint equations are always nonlinear. This enables the elimination of the constraints by a constant projection and allows for the system level ODE description. 
In the general case of nonlinear constraints, $\kernel$ depends on the configuration of the system. Therefore, it is impossible to define a constant mode set which always meets the constraint equations in the general case. 
As an example, consider the FFRF-based CMS method, where nonlinear constraint equations follow due to the co-rotated flexible coordinates. 
In the GCMS framework, on the contrary, the total displacement is a linear function of the DOFs. 
A fundamental joint like a spherical joint between different points can therefore be expressed as linear relations of the form
\begin{equation}
\left(\phigcms\qgcms\right)_{ij} - \left(\phigcms\qgcms\right)_{kl} - b = 0,
\end{equation}
where DOF $j$ of body $i$ has to be equal to DOF $l$ of body $k$ and an offset $b$ can be present, e.g., in order to realize a connection to a fixed frame. 
This is very straightforward for spherical joints, and it is important to notice that other joint types can be represented by combining multiple of these spherical joints. Revolute joints, for example, can be defined analogously by two or more spherical joints on the rotational axis. 
This shows that the restriction to linear constraint relations does no impose a strong restriction on general applicability of the proposed combined approach.

As intended, the full constraint equations are linear functions of the generalized coordinates in this case,
\begin{equation}
\C(\qgcms) = \C_{\rm lin} \qgcms - \mathbf b ,
\end{equation}
where $\C_{\rm lin}$ is a constant matrix. Employing a GMP-like projection,
\begin{equation}
\qgcms = \qgcms_0 + \kernel\qgmp,
\label{eq:qgcmsifqgmp}
\end{equation}
with a reference position $\qgcms_0$ such that,
\begin{equation}
\C_{\rm lin} \qgcms_0-\mathbf{b}=\mathbf{0}
\end{equation}
a constant kernel $\kernel$ can now be defined as
\begin{equation}
\kernel = \left\{\qgcms \in \realn , \C_{\rm lin} \qgcms = \mathbf{0} \right\} = \mathrm{const}.
\end{equation}
The construction of the constraint elimination space requires the computation of the null space of the constraint equations. 
Over the years, many different numerical algorithms have been developed which can be exploited for computing this kernel efficiently. In the present work, a singular value decomposition approach is utilized, which is readily available in many numerical libraries as, e.g.,~LAPACK~\cite{lapack}. The singular value decomposition of the constraint jacobian is given by:
\begin{equation}
\C_{\rm lin} = U\Xi V,
\end{equation}
where $\Xi \in \mathbb{R}^{n_c\times n}$ is a rectangular matrix containing the singular values for the constraint jacobian. Assuming $\C_{\rm lin}$ of full rank (which is typically true for FMBS), the null space can be constructed by selecting the trailing $n-n_c$ columns of $V$:
\begin{equation}
\kernel = \begin{bmatrix} V_{n_c+1} & ... & V_{n}\end{bmatrix}.
\end{equation}
Even though this full singular value decomposition is a relatively expensive operation, it only needs to be performed once and the number of DOFs in FMBS models is typically sufficiently limited (due to the initial GCMS reduction) that this poses little computational burden overall. However, due to the initial GCMS reduction, there is little sparsity which can be exploited in order to speed up these computations. As is discussed in Section~\ref{sec-reduction_space}, using different reduction spaces can also circumvent the explicit computation of this null-space.     

This constraint elimination leads to the first level of model reduction which can be achieved for a GCMS model:
\begin{eqnarray}
\rhosl &=& \qgcms_0, \\
\phisl &=& \kernel
\end{eqnarray}
The stored quantities for this approach comprise the undeformed configurations~$\rhosl$, the flexible modes~$\phisl$, the reduced mass matrix~$\Mgmp$.
The inertial terms can be evaluated directly on tin the reduced space, but at this point the evaluation of the nonlinear internal forces still requires a back-transformation to the original GMCS space. 

However, this first approach only allows limited computational gains and the following two sections will discuss how this advantage can be further extended by adding additional layers of model reduction. 



% -----------------------------------------------------------

\section{{System-level reduction space}} 
\label{sec-reduction_space}


At this point only the constraints are eliminated from the set of DOFs, which can still leave a considerable number of DOFs. Moreover, in order to obtain this constraint elimination, the null space for the constraint jacobian has to be computed explicitly, which can be a relatively expensive operation. 

In this section we discuss the addition of an additional model order reduction layer in which only the most important system level dynamics are maintained for the motion space. It is important to highlight that, in contrast to what has been proposed for GMP before \cite{Bruls_IJNME2008, NaetsHeirman2011}, a constant reduction space is proposed in this work in order to prevent the addition of involved inertial forces. This is again possible because of the constant constraint jacobian obtained for the GCMS model. 
  
This work will explore two different approaches to the reduction space generation, an a-priori system level modal reduction space and a posterior training based POD reduction. The generation of these reduction space can also be exploited to circumvent the explicit generation of the constraint null-space, as will be discussed in the following sub-sections. 

\subsection{System-level modal reduction space}
\label{sec:sl_modal}
% A-priori model reduction: compute a number of reference configurations, compute the eigenmodes. Keep the rigid body modes explicitly.
A first reduction strategy which is investigated for the system level reduction of a GCMS model is an a-priori strategy where the dominant system modes are retained. In many applications it is desirable to have reduction space which can be obtained without performing any training, in contrast to many methods proposed for nonlinear model reduction in literature \cite{DEIM,ECSW}, as the training is often considered prohibitively expensive. However, due to the very large rotations present in FMBS, simple extension strategies of linear reduction spaces, like the addition of modal derivatives \cite{Idelsohn1985,RutzmoserTiso?} do not offer a valid alternative either. 

In this work we therefore propose an a-priori strategy similar to the reduction space generation employed for GMP \cite{Bruls_IJNME2008}. In this approach the dynamics of the system are evaluated for several reference configurations and only the lowest frequency modes are retained. Rather than interpolating between these spaces as is the case for GMP, these modes are grouped together in order to obtain a constant reduction space. 

This reduction space consists of two main contributions: the system level rigid body modes $\phisl_r$ and the flexible deformation modes $\phisl_f$: 
\begin{itemize}
\item In order to prevent locking of the rigid body motion, the rigid body motion and flexible motion are treated separately. The rigid body motion is treated in such a way that any rigid configuration of the system can be represented exactly. 
In order to compute the system level rigid body modes, the constraint jacobian is partitioned into the contributions belonging to the translational, rotational and flexible GMCS DOFs (in correspondence with Eq.~\eqref{eq:phigcms}):
\begin{equation}
\C_{\rm lin} = \begin{bmatrix} \C_{\rm lin,t} & \C_{\rm lin,r} & \C_{\rm lin,f}\end{bmatrix}.
\end{equation}
Using this decomposition, the rigid body modes can be obtained by considering the null space of the constraint equations pertaining to the translational and rotational DOFs: 
\begin{eqnarray}
\phisl_{rr} &= ker\left(\begin{bmatrix} \C_{\rm lin,t} & \C_{\rm lin,r}\end{bmatrix}\right) \\
& = \left\{\q \in \realn , \begin{bmatrix} \C_{\rm lin,t} & \C_{\rm lin,r}\end{bmatrix}\q = \mathbf{0} \right\}.
\end{eqnarray}
Computing this null-space according to the SVD procedure outlined in Section~\label{sec:constraint_elimination} is typically a relatively low-cost operation in this case as both the number of constraints and rigid body DOFs are limited (rarely exceeding thousand DOFs) which is easily tractable on modern computers. If the system is rigidly overconstrained, it might be necessary to adjust the tolerance for the singular value cut-off. Such a case is shown in Sec.~\ref{sec-numerical_example}.
The full system rigid body modes are then obtained as:
\begin{equation}
\phisl_r = \begin{bmatrix} \phisl_{rr} \\ 0_{n_f\times n_r}\end{bmatrix}.
\end{equation}
Similar as for the unreduced GCMS model it is interesting to notice that these are not purely rigid motion modes, as the contribution from the rotational modes also introduces some shear deformation. 
 
\item The system level flexible deformation modes are treated in a more approximate manner. For the construction of the system level flexible reduction basis a number $n_{ref}$ of undeformed reference configurations $\rho(\eta^{sl}_{r,i})$ (with $i=1,\ldots,n_{ref}$) are visited with:
\begin{equation}
\rho(\eta^{sl}_{r,i}) = \left\{\q=\phisl_r \eta^{sl}_{r,i} , \Eint\left(\q\right) = \mathbf{0} \right\}.
\end{equation}
For each of these configurations $i$, the free-free eigenmodes $V^{sl}_i$ are computed:
\begin{equation}
\left(-\Msl\omega_i^2 + \kernel \frac{\partial \Fint}{\partial \q}\Bigr|_{\rho(\eta^{sl}_{r,i})}\kernel\right)V^{sl}_i = 0
\label{eq:eigenmodes_cons_elim}
\end{equation}
In order to compute the modes in this fashion, the constraint kernel $\kernel$ has to be computed explicitly. This can be circumvented by solving the constrained eigenvalue problem instead:
\begin{equation}
\left(-\begin{bmatrix}\Msl & 0 \\ 0 & 0\end{bmatrix}\omega_i^2 
+ \begin{bmatrix}\frac{\partial \Fint}{\partial \q}\Bigr|_{\rho(\eta^{sl}_{r,i})} & C^T \\ C & 0 \end{bmatrix}
\right)\begin{bmatrix} V^{sl}_i \\ V^{sl,\lambda}_i \end{bmatrix} = 0
\label{eq:eigenmodes_cons}
\end{equation}
Depending on the specific model it can be more convenient/efficient to use the formulation from Eq.~\eqref{eq:eigenmodes_cons_elim} or Eq.~\eqref{eq:eigenmodes_cons}. 
From this modal space for each reference configuration $i$, the first $n_m$ non-rigid modes are retained for each configuration $V^{sl}_{i,1:n_m}$. 
For each mode number these are assembled in a large matrix and an SVD is performed to extract the dominant components for each mode number:
\begin{equation}
\phisl_f = \begin{bmatrix} svd\left(\left[V^{sl}_{1,1},\ldots,V^{sl}_{n_{ref},1}\right]\right), & \ldots, & svd\left(\left[V^{sl}_{n_{ref},n_m},\ldots,V^{sl}_{n_{ref},n_m}\right]\right) \end{bmatrix}.
\end{equation}
Depending on the aim of the model reduction, these singular value decompositions can be truncated based on the number of modes or based on the singular values, which leads to a model of a-priori unknown size. The approach we propose in this work is to select the basis both on a maximum number of modes and a cut-off on the singular values. This approach is outlined in Algo.~\ref{algo:ModalBasisSelection}.
% Start algorithmic description:
\begin{algorithm}
\caption{System-level flexible modal basis construction and selection.}
\label{algo:ModalBasisSelection}
\begin{algorithmic}[1]
%\Procedure{CH\textendash Election}{}
% \IN $n_r$, $\sigma_c$, $\rho(\eta^{sl}_{r,i})$
\STATE Construct the full modal space:
\STATE $V^t_{1:n_r} = [\quad]$
\FOR{$i = 1:n_{ref}$}
\STATE $V^{sl}_{i} = eig\left(\kernel \frac{\partial \Fint}{\partial \q}\Bigr|_{\rho(\eta^{sl}_{r,i})}\kernel, \Msl\right)$
\FOR{$j = 1:n_r$}
\STATE $V^t_{j} = \left[V^t_{j}, V^{sl}_{i,j}\right]$
\ENDFOR
\ENDFOR
\STATE Select from full modal space:
\STATE $\phisl_f = [\quad]$, $cnt = 0$, $j=1$
\WHILE{$cnt < n_r$}
\STATE $[U,\Sigma,V] = svd(V^t_j)$
\STATE $k = 1$
\WHILE {$\Sigma(k)/\Sigma(1)\geq\sigma_c$ AND $cnt<n_r$ AND $k\leq n_{ref}$}
\STATE $\phisl_f = [\phisl_f, U_k]$
\STATE $cnt = cnt+1$, $k = k+1$
\ENDWHILE
\STATE $j=j+1$
\ENDWHILE
\end{algorithmic}
\end{algorithm}
As is known for regular modal reduction bases, they offer little rigorous error estimation and the same is true for the approach proposed here. Both the size of the reduced model $n_r$ and the cut-off singular value $\sigma_c$ will have an impact on the accuracy of the final ROM. However it is difficult to predict a reliable choice for either in advance. The evolution of the error for two cases are considered in Sec.~\ref{sec-numerical_example} and could serve a rough guidelines for a practical selection. 
Future research will be conducted on how schemes based on input/attachment modes, Krylov bases and balanced truncation \cite{Witteveen} can be exploited in this context. 
\end{itemize}
The rigid and flexible mode-set are then concatenated and a final SVD is performed in order to prevent linear dependencies between the different mode-sets\footnote{As no tracking of e.g. mode-switching is performed, it could happen that a single modal contribution manifests itself in multiple sets which could lead to linear dependencies.}:
\begin{eqnarray}
U\Xi V = svd(\begin{bmatrix} \phisl_r & \phisl_f\end{bmatrix}), \\
\phisl = \left\{U_i, \Xi_{ii} > 0\right\}. 
\end{eqnarray}
In order to further improve online computational efficiency, the SVD can be mass weighted in order to obtain a mass-orthogonal reduction space. 
Depending on the application it can be relatively straightforward to find a number of reference configuration (in the case of a limited number of system-level rigid body modes) or this can become rather involved. However, we do not investigate the (minimal) selection of the reference configurations further in this work. 


\subsection{Proper orthogonal reduction space}

The second approach we consider follows a classical nonlinear model order reduction approach where a (range of) training simulation(s) is performed from which the dominant motion vectors are extracted. More particularly teh proper orthogonal decomposition (POD) is employed in this work \cite{POD}.

Prior to the model reduction a number $m_t$ of training runs is performed. The response space matrix for each run $i$ is stored in $\mathcal{Q}_i$, and the responses of all training runs are grouped together in $\mathcal{Q}_t$:
\begin{equation}
\mathcal{Q}_t = \begin{bmatrix} \mathcal{Q}_1, & \ldots ,& \mathcal{Q}_{m_t} \end{bmatrix}.
\end{equation}
Next an undeformed reference configuration $\rhosl$ is chosen for the model reduction. This can be extracted from the training dated or created separately. This vector is then subtracted from $\mathcal{Q}_t$ in order to provide $\mathcal{Q}_t^*$, which contains the training displacements with respect to $\rhosl$.
In order to extract the dominant contributions in $\mathcal{Q}_t^*$, a singular value decomposition is performed:
\begin{equation}
U\Xi V = svd(\mathcal{Q}_t^*).
\end{equation}
The system level reduction space $\phisl$ is then obtained by retaining the $m$ dominant contributions from this SVD:
\begin{equation}
\phisl = \left\{U_i, \Xi_{ii} > 0\right\}. 
\end{equation}
This approach has several drawbacks which might limit its practical applicability:
\begin{itemize}
\item The cost of evaluating the training simulations can be prohibitively large, especially if the actual range of the MOR has to be sufficiently large.
\item It cannot be guaranteed that e.g. an exact rigid motion of the system is possible as there is no explicit inclusion of the rigid body modes. 
This issue could potentially be solved by adding the rigid body modes from Section~\ref{sec:sl_modal} explicitly and orthogonalizing $\mathcal{Q}_t^*$ with respect to these modes prior to performing the SVD. This approach would allow for a general rigid body motion, with a trained flexible deformation. For the sake of brevity, this approach is not further examined in this work. 
\end{itemize}
An important advantage of this approach lies in the fact that it is not required to explicitly compute the kernel of the constraint jacobian. When the training simulations are performed with the regular DAE description of the model, the resulting response has to comply with the constraints as well. As the constraints are linear, any linear combination of this response space (like the obtained through the SVD) will also comply with the constraints.
The training based reduction should also be fully robust with respect to an over-constrained rigid body motion, which can prove convenient in many applications (as shown in Section~\ref{sec-numerical_example}). 




% -----------------------------------------------------------

\section{{Reduced evaluation of system level internal forces}} \label{sec-hyper_reduction}
% Discuss the reduced cost evaluation of the internal forces for the proposed approach.

With the reduction spaces defined in the previous section, it is straightforward to evaluate the inertial forces for the reduced order model by directly projecting the GCMS mass matrix:
\begin{equation}
F_{iner}^{sl} = {\phisl}^T\Mgcms\phisl\ddetasl = \Msl\ddetasl,
\end{equation}
and as discussed before, the joint equations vanish from the equations of motion due to the selection of the system level reduction space. 

% General idea:
However, the internal potential forces for a GMCS model $\Fint$ are strongly nonlinear, and can therefor not be directly projected in order to have a computational load independent of the original model size. In previous work the reduced DOFs $\eta$ were projected back onto the original DOFs $\q$ in order to evaluate the internal forces \cite{Humer}. This allows to exploit some of the benefits of the reduced order model, as the number of DOFs for which the equations of motion need to be solved is reduced and there is greater flexibility due to the ODE structure. However, for a GCMS model with a large number of flexible DOFs, this can still lead to an expensive scheme due to the cost of evaluating (and differentiating) the nonlinear internal forces. 
In this section we will therefor investigate how the internal forces for the system-level reduced order model can be expressed without performing the back-projection to the original GCMS coordinates.    


The first aspect to consider is the rotation matrix $\A$ (and $\Abd$). 
According to the literature for corotational finite elements, this rotation matrix can be determined in a number of ways. In this work we focus on the frame-by-three-nodes approach proposed in previous work on GCMS modeling \cite{Pechstein2013}, but similar schemes to the one proposed here can (easily) be devised for other approaches. In this approach, the body-attached rotation matrix for body $l$ is obtained from the global position of three nodes of this body. These nodal coordinates for body $l$ $d_A^l \in \mathbb{R}^9$ are extracted from the full system GCMS coordinate vector through a selection matrix $S_A^l \in \mathbb{R}^{9\times n}$ for body $i$:
\begin{equation}
d_A^l = S_A^l\qgcms,
\end{equation}
which enables a straightforward reduction:
\begin{equation}
d_A^l \approx d_{Ar}^l = S_A^l\left(\rho+\phisl\eta\right) = d_{A\rho}^l + S_{Ar}^l\eta,
\end{equation}
which allows an evaluation of the body level rotation independent of the original number of DOFs through the reduced selection matrix $S_{Ar}^l$ for each body $l$:
\begin{equation}
\A = \A\left(d_{A\rho}^l+S_{Ar}^l\eta\right).
\end{equation} 
Also the derivatives of $\A$ can then be evaluated directly as a function of one of the system-level reduced DOFs $\eta_i$:
\begin{equation}
\frac{\partial \A}{\partial \eta_i} = \sum_{j=1}^9\frac{\partial \A}{\partial d_{A,j}^l}S_{A,ji}^l.
\label{eq:rotmat_red_der}
\end{equation}


We now define a selection matrix $P^{ij}\in \mathbb{R^{3 \times 3 }}$ with:
\begin{eqnarray}
P^{ij}_{kl} &= 0, k\neq i \text{ or } l\neq j, \\
P^{ij}_{kl} &= 1, k= i \text{ and } l= j,
\end{eqnarray}
which allows to select the different elements of the rotation matrix $\A$. Similar to $\A$ and $\Abd$, these selection matrices can also be structured for the full body in $P^{ij}_{bd}$. With these matrices, the body rotation matrix can be expressed as \footnote{With notation $\sum_{a,\ldots,d=1}^n = \sum_{a=1}^n\ldots\sum_{d=1}^n$}:
\begin{equation}
\Abd = \sum_{i,j=1}^{3}P^{ij}_{bd}\A_{ij},
\end{equation}
in terms of the different elements of the rotation matrix $\A$. 

With this description for the body rotation, the internal forces acting on the reduced order coordinates $\Fint^{l,\etasl}$ for body $l$ can be expressed as:
\begin{eqnarray}
\Fint^{l,\etasl} & = & {\phisl}^T\Fint^l(\rho + \phisl\etasl), \\
&=& \sum_{h,i,j,k=1}^{3} {\phisl}^TP^{hi}_{bd}\Kred P^{jk}_{bd}(\rho+\phisl\etasl)\A_{ih}\A_{jk} \nonumber \\
&& -\sum_{h,i=1}^{3} {\phisl}^TP^{hi}_{bd}\Kred^*\A_{ih} \nonumber \\
&& + \sum_{h,i,j,k=1}^{3}\left[\left(\rho+\phisl\etasl\right)^TP^{hi}_{bd}\Kred P^{jk}_{bd}\left(\rho+\phisl\etasl\right)\frac{\partial\A_{ih}}{\partial\etasl_i}\A_{jk}\right]_{i=1:n_{sl}} \nonumber \\
&& - \sum_{h,i=1}^{3}\left[\left(\rho+\phisl\etasl\right)^TP^{hi}_{bd}\Kred^*\frac{\partial\A_{ih}}{\partial\etasl_i}\right]_{i=1:n_{sl}},
\end{eqnarray}
where the rotation matrix derivatives are evaluated according to Eq.~\eqref{eq:rotmat_red_der}.
This is a rather elaborate equation for the internal forces, but all terms dependent on the original model size can be projected onto the system-level reduced model before simulation, such that the internal forces for a body can be written as:
\begin{eqnarray}
\Fint^{l,\etasl} &=& \sum_{h,i,j,k=1}^{3} \left(\Ksl^{hijk}\etasl + \Ksl^{\rho,hijk}\right)\A_{ih}\A_{jk} -\sum_{h,i=1}^{3} \Ksl^{*,hi}\A_{ih} \nonumber \\
&& + \sum_{h,i,j,k=1}^{3}\left[\left(\etasl^T\Ksl^{hijk}\etasl+2\etasl^T\Ksl^{\rho,hijk}+\Ksl^{\rho\rho,hijk}\right)\frac{\partial\A_{ih}}{\partial\etasl_i}\A_{jk}\right]_{i=1:n_{sl}} \nonumber \\
&& - \sum_{h,i=1}^{3}\left[\left(\etasl^T\Ksl^{*,hi} + \Ksl^{*\rho,hi}\right)\frac{\partial\A_{ih}}{\partial\etasl_i}\right]_{i=1:n_{sl}},
\end{eqnarray}
with
\begin{eqnarray}
\Ksl^{hijk} &=& {\phisl}^TP^{hi}_{bd}\Kred P^{jk}_{bd}\phisl, \\
\Ksl^{\rho,hijk} &=& {\phisl}^TP^{hi}_{bd}\Kred P^{jk}_{bd}\rho, \\
\Ksl^{\rho\rho,hijk} &=& \rho^TP^{hi}_{bd}\Kred P^{jk}_{bd}\rho, \\
\Ksl^{*,hi} &=& {\phisl}^TP^{hi}_{bd}\Kred^*, \\
\Ksl^{*\rho,hi} &=& \rho^TP^{hi}_{bd}\Kred^*,
\end{eqnarray}
with $\Kred$ and $\Kred^*$ as given in respectively Eq.~\eqref{eq:Kred} and Eq.~\eqref{eq:Kred*}. 
It is not necessary store all of these matrices as $\Kred$ is symmetric and therefor many of these matrices are symmetric with respect to $h,i,j,k$. These matrices need to be stored for every body in the system, but their sizes are independent off the original body size, leading to an internal force evaluation cost independent of the original model size. 
 
In order to obtain the full internal force vector for the system level reduced order model, the contributions from all bodies need to be summed together:
\begin{equation}
\Fint^{\etasl} = \sum_{l=1}^{n_{body}}\Fint^{l,\etasl}. 
\end{equation}
With this formulation, the computational load becomes independent of the original number of DOFs and scales with the number of bodies and system-level modes. In practice this implies that the proposed approach scales favorably if the GCMS model consisted of a limited number of bodies with many flexible DOFs for each body and that a regular back-projection might be preferable if a large number of bodies with only a few flexible DOFs are present. 
 
The main benefit of this approach over other more general reduction schemes is the fact that besides the projection, no further approximations are made on the equations of motion for the system. The nonlinear internal forces are evaluated exactly for the system level reduction basis. 



% -----------------------------------------------------------
\section{{Numerical validation}}
\label{sec-numerical_example}
% NEEDS TO BE UPDATED!
% Re-run this example in matlab code? -> ask FE definition files to Alex? (and then just use tetra's from Nastran)

% Numerical examples:
% - 1 rigid dof, non-redundant rigid constraints
% - 1 rigid dof, redundant rigid constraints
% - 4-bar, 2 closures with and without training in the two configurations

In this section we perform two validation cases for the system-level model order reduction scheme based on a GCMS flexible multibody model. In both cases the reduction performance of the two presented schemes is compared. The first system is a double pendulum which demonstrates the performance in the case of multiple rigid system DOFs. This is a model structure which cannot be handled adequately by the GMP approach, previously presented for system level model reduction for FMBS \cite{Bruls2008}. The second validation case is a six-bar mechanism. This is a particularly challenging case due to the overly constrained rigid motion and offers interesting potential for model reduction due to the strongly coupled motion between the components. 

This section is focused on the reduction in the number of DOFs and the resulting accuracy. Computational load is not discussed in detail as all implementations are in Matlab \cite{Matlab} which does not provide a reliable platform for absolute computational load evaluation. However, in all cases discussed here, the final computational load scaled well with the number of DOFs. Besides the lower computational load of evaluating the equations of motion, the smaller scale models also offered faster convergence during the iterative phase of the implicit time integration. 

% -------------------------------------------------------------
\subsection{Double pendulum}
The first model considered is a flexible double pendulum with a load applied at the free end. The overall motion of the double pendulum system is shown in Fig.~\ref{fig:DoublePendulum}. 
\begin{figure}[htp]
\centering
\includegraphics[bb= 1cm 1cm 14cm 4cm, clip, scale=1]{DoublePendulum.eps} 
\caption{Motion of the (unreduced) double pendulum model.}
\label{fig:DoublePendulum}
\end{figure}

\paragraph{Simulation setup}
% Description of bodies.
This system consists of only two bodies. The two bodies are modeled identically and are rectangular sections on which the same finite element discretization is performed with hexahedron elements in Nastran. The properties of each body are summarized in Tab.~\ref{tab:DoublePendulumProperties}. 
\begin{table}[htp]
\centering
\caption{Parameters for double pendulum body.} \label{tab:DoublePendulumProperties}
\vspace{1ex}
\begin{tabular}{l r}
\toprule
Length 	& $0.3m$	\\[0.2ex]
Height & $0.05m$	\\[0.2ex]
Width & $0.03m$	\\[0.2ex]
Youngs modulus & $2.1\cdot 10^7 N/m^2$	\\[0.2ex]
Poison Ration & $0.3$	\\[0.2ex]
Elements & $48$	\\[0.2ex] 
Nodes & $105$	\\[0.2ex]
Modes & $4$	\\[0.2ex] 
\bottomrule
\end{tabular}
\end{table}
With these parameters, each body has a total of forty-eight DOFs ($3$ rigid translation, $9$ rotation and $9\times 4$ flexible modes).
Only spherical joints are used in the system level model. One joint connects the center node of one end of the first body to the fixed world reference, and a second joint connects the center node of the second end of the first body to the center node of the free end of the second body. This leads to a system with six rigid DOFs and ninety free DOFs for the full system. 
A biased random force, with a component in the y- and z-direction, is applied at the free end of the second body. The course of this force is shown in Fig.~\ref{fig:DoublePendulumLoad}.
\begin{figure}[htp]
\centering
\includegraphics[scale=1]{DoublePendulumLoad.eps} 
\caption{Tip loads on double pendulum model.}
\label{fig:DoublePendulumLoad}
\end{figure}
The dynamic simulation is performed using a generalized alpha integrator with an exact Newton-Raphson solver\cite{BrulsGenAlpha}. The simulation is run for $0.5sec$ with a timestep of $\Delta t = 10^{-3}sec$ and the numerical damping factor set to $\rho = 0.1$ (no model damping is added). 

\paragraph{Model reduction setup}
First we investigate the selection of the reduction space for the double pendulum model. For the POD space, the previously described simulation is run first in order to provide a training trajectory for the system. The singular value decomposition performed on this system leads to the singular values (normalized with respect to the largest singular value) shown in Fig.~\ref{fig:DoublePendulumSigmas}.
\begin{figure}[htp]
\centering
\includegraphics[scale=1]{DoublePendulumPODSigmas.eps} 
\caption{Singular values for double pendulum response.}
\label{fig:DoublePendulumSigmas}
\end{figure}
It is interesting to notice that even though the system experiences a broad excitation,  the singular values still show a fast decay, indicating a considerable reduction potential. The final six singular values equal machine precision as these modes cannot be excited due to the constraints (only ninety free DOFs remain as discussed before). 

Secondly, we consider the construction of the modal reduction space. Many possibilities exist to select the undeformed configurations for which the modal space is constructed. For this numerical validation we employ the initial GCMS simulation results to extract a set of reference configurations (we choose every fiftieth response vector). For each of these training responses $\q_{tr}$ we compute the closest undeformed configuration $\q_{ref}$ by solving a constrained optimization problem:
\begin{eqnarray}
min & \left|\q_{ref}-\q_{tr}\right|_2 \\
s.t & F_{int}(\q_{ref}) = 0, \\
& C(\q_{ref}) = 0.
\end{eqnarray}
This problem looks for the closest undeformed configuration which meets the constraints. It is interesting to highlight that we use the internal forces for the constraint instead of the internal energy, as the latter leads to very slow convergence near the solution due to the internal forces also approaching zero. 
For each of these reference configurations the first four system-level eigenmodes are computed and these are grouped together to perform the singular value decomposition, as outlined in Sec.~\ref{sec-reduction_space}. The (normalized) singular values for these four decompositions are shown in Fig.~\ref{fig:DoublePendulumModalSigmas}.
\begin{figure}[htp]
\centering
\includegraphics[scale=1]{DoublePendulumModalSigmas.eps} 
\caption{Singular values for the four first system level eigenmodes of the double pendulum.}
\label{fig:DoublePendulumModalSigmas}
\end{figure}
It is interesting to notice that these singular values decay rather slow, leading to the assumption that the reduction efficiency might be relatively low. 
The reduction space construction here is based on setting a cut-off tolerance for the SVD and a maximum number of reduced bases. In the following paragraph we investigate how the final accuracy evolves when adjusting these two parameters. 

\paragraph{Results}
A plot of the response for the unreduced and two reduced models is shown in Fig.~\ref{fig:DoublePendulumResponse}.
\begin{figure}[htp]
\centering
\includegraphics[scale=1]{DoublePendulumResponse.eps} 
\caption{Response of double pendulum model: comparison of unreduced response (GCMS) with POD system level reduced model ($n_r=28$) and modal system level reduced model ($n_r=63$ and $\sigma_c=10^{-2}$).}
\label{fig:DoublePendulumResponse}
\end{figure}
This figure clearly shows that even for this relatively small initial model, a considerable reduction in the number of DOFs is possible using either of the proposed approaches. The POD approach offers superior performance, but this is also expected as the case considered is fully reproductive.
It is now also interesting to investigate how the model error evolves for a varying size of the reduced order system level model. 
Fig.~\ref{fig:DoublePendulumPODError} shows the decay of the difference between the original GCMS and reduced order SL-GCMS model with the POD basis, for an increasing number of POD modes. 
\begin{figure}[h!]
\centering
\includegraphics[scale=1]{DoublePendulumPODError.eps} 
\caption{Convergence of reduced order model with POD for double pendulum.}
\label{fig:DoublePendulumPODError}
\end{figure}
This figure shows a very fast decay, indicating that a small basis is able to accurately capture the simulated behavior. This is particularly interesting as the nonlinearity in this case is considerable.
Fig.~\ref{fig:DoublePendulumModalError} shows the course of the reduction error for different parameters (reduced model size and cut-off singular value) for the SL-GCMS model with a modal reduction basis. 
\begin{figure}[htp]
\centering
\includegraphics[bb= 0cm 0cm 10cm 5cm, clip,scale=1]{DoublePendulumModError.eps} 
\caption{Convergence of reduced order model with POD for double pendulum.}
\label{fig:DoublePendulumModalError}
\end{figure}
This figure still shows convergence as the ROM size increases, but the convergence is much slower than in the POD case. Moreover, also the cut-off singular value has an important impact on the accuracy for the larger ROMs. The observed behavior at $n_r=63$ can be explained intuitively. For very high cut-off, only a single mode is selected from each set, but this mode is not capable of capturing that full modal behavior accurately. However if the cut-off is selected too low, too much emphasis is put on the first few modes, again leading to a limited accuracy. 
For the given simulation case, the POD approach clearly provides a higher reduction efficiency than the modal approach. This is to be expected as the POD space is specifically tailored to this simulation case. However, both approaches demonstrate good accuracy for (considerably) smaller models, even in the case of this simple unreduced reference model. The POD SL-GCMS model is smaller, but less robust with respect to varying inputs and requires a full reference simulation. The modal SL-GCMS on the other hand should be relatively robust and the reduction space construction can be performed in parallel, leading to low overall simulation time, but does require a larger model for good accuracy. 

% -------------------------------------------------------------
\subsection{Five-bar mechanism}
A five-bar mechanism is investigated because there is a stronger coupling between the different components than for the pendulum model, which is expected (and shown) to be even more beneficial for model reduction. This model also proposes the additional difficulty that the rigid body motion is overly constrained, which makes it interesting to investigate how this can be handles in the POD and modal reduction space. The simulated motion of the system is shown in Fig.~\ref{fig:FiveBar}.
\begin{figure}[htp]
\centering
\includegraphics[bb= 1cm 1cm 14cm 4cm, clip, scale=1]{FiveBar.eps} 
\caption{Motion of the (unreduced) five-bar mechanism.}
\label{fig:FiveBar}
\end{figure}

\paragraph{Simulation setup}
% Description of bodies.
This system consists of four flexible bodies and the rigid ground. The three parallel bodies (body 1-3) are modeled identically and are rectangular sections on which the same finite element discretization is performed with hexahedron elements in Nastran. The fourth body (body 4) is longer and allows the connection of the three parallel bodies. The properties of these bodies are summarized in Tab.~\ref{tab:FiveBarProperties}. 
\begin{table}[htp]
\centering
\caption{Parameters for double pendulum body.} \label{tab:FiveBarProperties}
\vspace{1ex}
\begin{tabular}{l c c}
\toprule
 & Body 1-3 & Body 4 \\ 
\bottomrule
Length 	& $0.3m$	& $1m$\\[0.2ex]
Height & $0.05m$	& $0.05m$\\[0.2ex]
Width & $0.03m$	& $0.03m$\\[0.2ex]
Youngs modulus & $2.1\cdot 10^9 N/m^2$	& $2.1\cdot 10^9 N/m^2$\\[0.2ex]
Poison Ration & $0.3$	& $0.3$\\[0.2ex]
Elements & $1080$	& $3600$ \\[0.2ex] 
Nodes & $1519$	& $4949$\\[0.2ex]
Modes & $10$	& $10$ \\[0.2ex] 
\bottomrule
\end{tabular}
\end{table}
With these parameters, each body has a total of one-hundred-and-two DOFs ($3$ rigid translation, $9$ rotation and $9\times 10$ flexible modes).
Only spherical joints are used in the system level model. In this example each time two spherical joints are used to emulate the behavior of a two-bearing setup (typically modelled using a cylindrical joint). This joint setup is used to connect bodies one to three to the ground and to body four. It is interesting to notice that this setup would lead to a highly overconstrained rigid body system. For a flexible multibody model however, this is not an issue. The resulting system should have a single rigid DOF. 
 
A biased random force, with a component in the x- and y-direction, is applied at the right end of body four. The course of this force is shown in Fig.~\ref{fig:FiveBarLoad}.
\begin{figure}[htp]
\centering
\includegraphics[scale=1]{FiveBarLoad.eps} 
\caption{External force load on horizontal beam in five-bar model.}
\label{fig:FiveBarLoad}
\end{figure}
The dynamic simulation is again performed using a generalized alpha integrator with an exact Newton-Raphson solver\cite{BrulsGenAlpha}. The simulation is run for $0.5sec$ with a timestep of $\Delta t = 10^{-3}sec$ and the numerical damping factor set to $\rho = 0.1$ (no model damping is added). 

\paragraph{Model reduction setup}
We again begin by investigating the behavior of the reduction space for the five-bar mechanism. For the POD space, the previously described simulation is run first in order to provide a training trajectory for the system. The singular value decomposition performed on this system leads to the singular values (normalized with respect to the largest singular value) shown in Fig.~\ref{fig:FiveBarSigmas}.
\begin{figure}[htp]
\centering
\includegraphics[scale=1]{FiveBarPODSigmas.eps} 
\caption{Singular values for the five-bar model response.}
\label{fig:FiveBarSigmas}
\end{figure}
The singular values again show a rapid decay as was the case with the double pendulum.  

Secondly, we consider the construction of the modal reduction space. We use the reference simulation again to set up fifty reference configurations, as outlined for the double pendulum, and use Algo.~\ref{algo:ModalBasisSelection} again to construct the actual system-level modal basis. 
In this case it is also particularly interesting to consider the selection of the rigid body modes, as these should be overconstrained through the joint selection. Fig.~\ref{fig:FiveBarRigidSigmas} shows the singular value decomposition for the joint equations in the translational and rotational DOFs of the GCMS model. 
\begin{figure}[htp]
\centering
\includegraphics[scale=1]{FiveBarRigidSigmas.eps} 
\caption{Singular values for joint jacobian in rigid motion DOFs for five-bar model.}
\label{fig:FiveBarRigidSigmas}
\end{figure}
This figure clearly shows a strong decay after the first twenty-three singular values which is in correspondence with one exactly rigid mode. 
For each of these reference configurations the first ten system-level eigenmodes are computed and these are grouped together to perform the singular value decomposition, as outlined in Sec.~\ref{sec-reduction_space}. The (normalized) singular values for the first four eigenmodes are shown in Fig.~\ref{fig:FiveBarModalSigmas}.
\begin{figure}[htp]
\centering
\includegraphics[scale=1]{FiveBarModalSigmas.eps} 
\caption{Singular values for the four first system level eigenmodes of the five-bar model.}
\label{fig:FiveBarModalSigmas}
\end{figure}
These singular values again exhibit a relatively slow decay. 


\paragraph{Results}
A plot of the response for the unreduced and two reduced models for the five-bar model is shown in Fig.~\ref{fig:FiveBarResponse}.
\begin{figure}[htp]
\centering
\includegraphics[scale=1]{FiveBarResponse.eps} 
\caption{Response of five bar model: comparison of unreduced response (GCMS) with POD system level reduced model ($n_r=28$) and modal system level reduced model ($n_r=63$ and $\sigma_c=10^{-2}$).}
\label{fig:FiveBarResponse}
\end{figure}
This figure clearly shows that even for this relatively small initial model, a considerable reduction in the number of DOFs is possible using either of the proposed approaches. The POD approach offers superior performance, but this is also expected as the case considered is fully reproductive.
It is now also interesting to investigate how the model error evolves for a varying size of the reduced order system level model. 
Fig.~\ref{fig:FiveBarPODError} shows the decay of the difference between the original GCMS and reduced order SL-GCMS model with the POD basis, for an increasing number of POD modes. 
\begin{figure}[h!]
\centering
\includegraphics[scale=1]{FiveBarPODError.eps} 
\caption{Convergence of reduced order model with POD for the five-bar model.}
\label{fig:FiveBarPODError}
\end{figure}
This figure shows a very fast decay, indicating that a small basis is able to accurately capture the simulated behavior. This is particularly interesting as the nonlinearity in this case is considerable.
Fig.~\ref{fig:FiveBarModalError} shows the course of the reduction error for different parameters (reduced model size and cut-off singular value) for the SL-GCMS model with a modal reduction basis. 
\begin{figure}[htp]
\centering
\includegraphics[bb= 0cm 0cm 10cm 5cm, clip,scale=1]{FiveBarModError.eps} 
\caption{Convergence of reduced order model with POD for the five-bar model.}
\label{fig:FiveBarModalError}
\end{figure}
Blablabla
% -----------------------------------------------------------
\section{Conclusion}
NEEDS TO BE UPDATED. 
In the present paper, a novel approach for an efficient simulation of flexible multibody systems based on modal reduction methods has been discussed. Combining the advantageous properties of the GCMS formulation with the ideas of the GMP approach for a system level reduction is a promising step towards a more efficient analysis and real-time simulations of complex problems.
The GCMS formulation provides the key ingredients required by a GMP approach in order to perform well. The linear configuration space of such absolute coordinate based approach significantly reduces the computational efforts compared to classical FFRF-based methods involving non-linear inertia terms. 
Abandoning a description of flexible displacements relative to the rigid body motion, the GCMS shape functions are constant for the entire configuration manifold, i.e., the possible configurations of a body in the course of motion. 
%Therefore, it is not necessary to perform some sort of interpolation of varying shape functions as it is usually the case in conventional GMP approaches. 
In contrast to the time dependent projection space of classical GMP approaches, which requires an expensive pre-computation and interpolation of several quantities, the properties of the GCMS formulation allow a constant projection when performing a system level reduction.
A second crucial feature enabling an efficient simulation is the simple linear structure of the algebraic constraint equations of many conventional kinematic pairs when employing absolute coordinates. 
As a consequence, the null space projection that is used in GMP to eliminate constraints from the equations of motion is also constant in practical real-world applications. 
As explained above, the approach discussed in the present paper only represents the first of several steps used in GMP for model order reduction.
The validity and efficiency have been demonstrated by means of a first numerical example which has already shown a significant speed-up without reducing the accuracy of the results.
In future work, further reduction steps promising an even more efficient combination of the GCMS and GMP approaches will be investigated. 
In addition to what has been mentioned above, the applicability of fast explicit time integration schemes will be studied since a constant number of operations is a key requirement of real-time systems.
%The present paper focuses on the first step 


% -----------------------------------------------------------
\section*{Acknowledgements}
This work benefits from the Belgian Programme on Interuniversity Attraction Poles, initiated by the Belgian Federal Science Policy Office (DYSCO). The IWT Flanders within the MODRIO project is also gratefully acknowledged for their support. The authors also acknowledge the support of ITEA2 through the MODRIO project. 
The research of Frank Naets is funded by a postdoctoral fellowship of the Fund for Scientific Research, Flanders (F.W.O).
%
The authors Humer and Gerstmayr have been supported by the Linz Center of Mechatronics (LCM) in the framework of the Austrian COMET-K2 programme.

%\label{sec:1}
%Text with citations \cite{RefB} and \cite{RefJ}.
%\subsection{Subsection title}
%\label{sec:2}
%as required. Don't forget to give each section
%and subsection a unique label (see Sect.~\ref{sec:1}).
%\paragraph{Paragraph headings} Use paragraph headings as needed.
%\begin{equation}
%a^2+b^2=c^2
%\end{equation}

%% For one-column wide figures use
%\begin{figure}
%% Use the relevant command to insert your figure file.
%% For example, with the graphicx package use
%  \includegraphics{example.eps}
%% figure caption is below the figure
%\caption{Please write your figure caption here}
%\label{fig:1}       % Give a unique label
%\end{figure}
%
%% For two-column wide figures use
%\begin{figure*}
%% Use the relevant command to insert your figure file.
%% For example, with the graphicx package use
%  \includegraphics[width=0.75\textwidth]{example.eps}
%% figure caption is below the figure
%\caption{Please write your figure caption here}
%\label{fig:2}       % Give a unique label
%\end{figure*}
%
%% For tables use
%\begin{table}
%% table caption is above the table
%\caption{Please write your table caption here}
%\label{tab:1}       % Give a unique label
%% For LaTeX tables use
%\begin{tabular}{lll}
%\hline\noalign{\smallskip}
%first & second & third  \\
%\noalign{\smallskip}\hline\noalign{\smallskip}
%number & number & number \\
%number & number & number \\
%\noalign{\smallskip}\hline
%\end{tabular}
%\end{table}


%\begin{acknowledgements}
%If you'd like to thank anyone, place your comments here
%and remove the percent signs.
%\end{acknowledgements}

% BibTeX users please use one of
%\bibliographystyle{spbasic}      % basic style, author-year citations
\bibliographystyle{spmpsci}      % mathematics and physical sciences
%\bibliographystyle{spphys}       % APS-like style for physics
\bibliography{bib}   % name your BibTeX data base


\end{document}
% end of file template.tex

