%------------------------------------------------------------
% Section on GCMS: Alex + Johannes
\section{{Generalized component mode synthesis}} \label{sec-gcms}
%++++++++++++++++++++++++++++++++++++++++
%some text here still copy/paste from ASME paper!!!
%++++++++++++++++++++++++++++++++++++++++

% I can probably keep this section as-is...

%General overview of GCMS (high level). 
%Current bottlenecks...
%\subsection{Modal projection}
%Explain 12+9Nm modes.
%\subsection{Equations of motion}
%Show reduced equations of motion. 

The generalized component mode synthesis (GCMS) is a reduction method for flexible multibody systems by means of a formulation based on inertial or absolute coordinates \cite{gerstmayr2006}. 
Such a formulation, which is also denoted as absolute coordinate formulation (ACF), can be regarded as a method dual to the FFRF in the sense that the complete set of coordinates is defined in the global frame rather than employing coordinates relative to a co-rotational frame.
%In what follows, the formulation is shortly revisited.

\subsection{Absolute coordinate formulation and modal reduction}

Following Pechstein et al.~\cite{pechstein2013}, we introduce a modal reduction of some coordinates~$\q$, e.g., nodal displacements of solid finite element models,
\be
  \q \approx \phigcms \qgcms. 
\ee
To enable a computationally efficient analysis, the reduced set of coordinates $\qgcms$ needs to be of much smaller dimensionality as compared to the original coordinates~$\q$ but still describe the body's dynamic behavior sufficiently accurate.
%The reduction matrix $\phigcms$ is obtained from appropriate mode shapes, e.g., body-local eigenmodes or static modes, by a particular transformation that accounts for the body's rigid body motion. 
%For further details on the structure of~$\phigcms$, we again refer to Pechstein et al.~\cite{pechstein2013}.
We emphasize the crucial idea of the reduction being constructed such that the total displacement $\u$ of any point $\x$ of a body is linearly related to the reduced coordinates $\qgcms$ via shape functions $\Ngcms$:
\begin{equation} \label{eq:lin}
\u(\x) = \Ngcms (\x) \qgcms .
\end{equation}
In the GCMS formulation, one single set of coordinates $\qgcms$ describes the entire deformation of a body, i.e., rigid body translation and rotation as well as flexible deformation. 
Such an approach differs significantly from the classical FFRF-based CMS method, which uses a different set of coordinates for the rigid body motion and the flexible part of displacements.

Applying modal reduction to the ACF, the reduction matrix $\phigcms$ is obtained from appropriate mode shapes, e.g., body-local eigenmodes or static modes, by a transformation that accounts for the rotation of a body. 
Each standard mode shape is converted into nine corresponding generalized mode shapes.
%A modal reduction method can be applied to the absolute coordinate formulation by transforming each standard mode shape into nine corresponding generalized mode shapes. 
These nine shape functions---resulting from the nine components of the rotation tensor---relate nine associated generalized coordinates to the small flexible displacements of an arbitrarily rotated body.
The underlying rigid body motion, represented by a rotation tensor~$\A$ and a rigid body translation $\ut$ is used for simplifying the computation of the elastic forces of the body.
The rotation tensor $\A$ is obtained by orthogonalization of position vectors of three referential nodes in the underlying finite element mesh. The reduction matrix can then be written as:
\begin{equation}
\phigcms = \begin{bmatrix} \phigcms_t & \phigcms_r & \phigcms_f\end{bmatrix},
\label{eq:phigcms}
\end{equation}
where $\phigcms_t$ are three translation modes for a body, $\phigcms_r$ are nine rotational modes and $\phigcms_f$ are the flexible deformation modes. 
For more details on the construction of reduction matrix $\phigcms$, which links the finite element shape functions $\mathbf N^{\rm FE}(\x)$ to those of the GCMS method, $\Ngcms(\x) = \mathbf N^{\rm FE}(\x) \phigcms$, we again refer to Pechstein et al.~\cite{pechstein2013}.

Remarkably, the coordinates that describe the rigid body rotation of each body also contain stretch and shear deformation if no further constraints are imposed.
A coordinate vector $\qrigid$ needs to be computed, which represents both the translational part and the rotation $\ur = \A \x$ contained in the total displacement, i.e.,
\begin{equation}
\ut(\x) + \ur(\x) = \Ngcms(\x) \qrigid .
\end{equation}
Consequently, the coordinate vector corresponding to the flexible displacements $\uf$ is immediately obtained as the difference $\qflex = \qgcms - \qrigid$ such that
\begin{equation}
\uf(\x) = \Ngcms (\x) \qflex .
\end{equation}
Having the relation of displacements and coordinates at hand enables a continuum mechanics based derivation of the equations of motion, see \cite{gerstmayr2006, pechstein2013}.

%
%++++++++++++++++++++++++++++++++++++++++++++++++++++++++++
\subsection{Modally reduced equations of motion}

The modally reduced equations of motion are derived by starting from a Langrangian description of the system (semi-disretized in space):
\begin{equation}
\mathcal{L} = \Einer - \Eint + \lambda^T\C - \Wext,
\end{equation}
where $\Einer$ represents the kinetic energy, $\Eint$ the internal potential strain energy, $\C$ is a vector with the constraint equations and $\lambda$ are the corresponding Lagrange multipliers and $\Wext$ is the work performed by external forces. The equations of motion are then obtained by applying Hamilton's principle for the GCMS DOFs and Lagrange multipliers:
\begin{eqnarray}
\frac{d}{dt}\left(\frac{\partial\mathcal{L}}{\partial \dqgcms}\right) - \frac{\partial\mathcal{L}}{\partial \qgcms} &= 0, \\
\C & = 0. 
\end{eqnarray}
This can be rewritten as (assuming only holonomic constraints):
\begin{eqnarray}
\Finer + \Fint + \left(\frac{\partial \C}{\partial \qgcms}\right)^T\lambda &= \Fext, \\
\C & = 0. 
\end{eqnarray}
In the following paragraph we briefly summarize how the different contributions in this equation are computed for a GCMS model description.

The inertial forces are easy to obtain (in contrast to a FFR model approach) due to the global nature of the DOFs, as discussed in previous works \cite{pechstein2013} the inertial forces $\Finer$ are obtained as:
\begin{equation}
\Finer = {\phigcms}^T\Mfe\phigcms\ddqgcms = \Mgcms\ddqgcms.
\end{equation}
External forces defined in the global reference frame are equally straightforward to project:
\begin{equation}
\Fext = {\phigcms}^T\Fext^{FE}. 
\end{equation}

However, in the case of the GCMS model description, the internal forces $\Fint$ are relatively involved. For this approach the internal potential strain energy $\Eint$ is obtained as:
\begin{equation}
  \Eint = \frac{1}{2} \left(\At\phigcms\qgcms -u_0\right)^T\Kfe\left(\At\phigcms\qgcms -u_0\right) ,
	\label{eq:w_int2}
\end{equation}
where the stiffness matrix $\Kfe$ is the original finite element stiffness matrix, $u_0$ are the undeformed coordinates expressed in the body axis system and $\At$ denotes a block-diagonal matrix of the size $n \times n$with the three-by-three rotation matrix $\A$ on its diagonal. This rotation matrix can be determined in a number of ways, analogously to the techniques developed for corotational finite elements \cite{Felippa}.  Due to the construction of the of the GCMS reduction space, it is possible to write \cite{gerstmayr2006}:
\begin{equation}
 \At\phigcms\qgcms = \phigcms\Abd\qgcms,
\end{equation}
with $\Abd$ a block-diagonal matrix of the reduced model size $n_{GCMS} \times n_{GCMS}$ with three-by-three rotation matrices $\A$ on its diagonal. The internal energy can then be written as:
\begin{eqnarray}
  \Eint &= \frac{1}{2} \left(\phigcms\Abd\qgcms -u_0\right)^T\Kfe\left(\phigcms\Abd\qgcms -u_0\right) , \\
	&= \frac{1}{2} \left(\phigcms\Abd\qgcms\right)^T\Kfe\left(\phigcms\Abd\qgcms\right) \nonumber \\
	& \qquad - \left(\phigcms\Abd\qgcms\right)^T\Kfe u_0+\frac{1}{2}u_0^T\Kfe u_0 ,\\
	&= \frac{1}{2} \left(\Abd\qgcms\right)^T\Kred\left(\Abd\qgcms\right) - \left(\Abd\qgcms\right)^T\Kred^*+\Eint^0,
\end{eqnarray}
with 
\begin{eqnarray}
\Kred &=& \left(\phigcms\right)^T\Kfe\phigcms, \label{eq:Kred}\\
\Kred^* &=& \left(\phigcms\right)^T\Kfe u_0. \label{eq:Kred*}
\end{eqnarray}
From this internal energy the internal forces acting on the GCMS DOFs can be computed as:
\begin{eqnarray}
\Fint & = & \frac{\partial \Eint}{\partial \qgcms}, \\
&=& \Abd^T\left(\Kred\Abd\qgcms-\Kred^*\right) \nonumber \\
&& + \left[\left(\frac{\partial\Abd}{\partial\qgcms_i}\qgcms\right)^T\left(\Kred\Abd\qgcms-\Kred^*\right)\right]_{i=1:n_{GCMS}}
\end{eqnarray}
with the notation:
\begin{equation}
\left[\frac{\partial f}{\partial q_i}K\right]_{i=1:n} = \begin{bmatrix} \frac{\partial f}{\partial q_1}K \\ \ldots \\ \frac{\partial f}{\partial q_n}K\end{bmatrix}. 
\end{equation}
Upon a closer inspection, one recognizes that the evaluation of these equations requires little computational expense, see \cite{pechstein2013} for further details concerning the implementation.
Although the number of degrees of freedom considerably exceeds that of the corresponding FFRF-based CMS method, the efficiency of the methods have proven to be comparable (or better) in industrial applications~\cite{pechstein2013}.
While the FFRF-based CMS method features a smaller number of generalized coordinates using the same number of flexible mode shapes, a non-constant mass matrix and gyroscopic inertial need to be evaluated in each computation step, which usually negates the advantage obtained by the smaller system size.

The treatment of the joint equations is treated in more detail in the following section. 

